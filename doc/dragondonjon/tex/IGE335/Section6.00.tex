\section{THE DRAGON PACKAGE}\label{sect:DragonExportPackage}

The following files are required to install DRAGON:

\begin{verbatim}
Version5_nnn.tgz
libraries_nnn.tgz
\end{verbatim}

\vskip 0.08cm

Information is recovered from the two archives using
\begin{verbatim}
tar xvfz Version5_nnn.tgz
tar xvfz libraries_nnn.tgz
\end{verbatim}

\noindent The {\tt tar xvfz} operations will create two directories named {\tt Version5} and {\tt
libraries} made of various components (files and directories), as shown in \Fig{tree2}.

\begin{figure}[h!]  
\begin{center} 
\epsfxsize=10cm \centerline{ \epsffile{distributionTree.eps}}
\parbox{16cm}{\caption{Distribution content.}\label{fig:tree2}}    
\end{center}  
\end{figure}

Directory {\tt Version5} contains the information required to install and configure
DRAGON. It is a copy of the {\sl Subversion Working Copy} used by the developers of the
code. Inside {\tt Version5} is a file named {\tt readme} that contain the
information required to configure DRAGON on your system. This configuration
process has the effect to add a few directories and binary files to the
{\tt Version5} directory.

\vskip 0.08cm

On Windows, Version5 components an be build in two possible ways:
\begin{itemize}
\item  A native build is possible, provided that {\sl Microsoft Visual Studio} and {\sl Intel Visual Fortran} are available.
A script named {\tt instver5.bat} is available to create executables ({\tt .exe}) from a MS-DOS command
window. Follow the following steps:
\begin{itemize}
\item Click {\sl Start}, point to {\sl All Programs}, {\sl Intel(R) Software Development Tools}, {\sl Intel(R) Visual Fortran
Compiler Professional} and finally {\sl Fortran Build Environment for applications running on IA-32}. The Visual Studio Command Prompt
automatically sets the correct paths of both C and Fortran compilers, together with their associated libraries.
\item In the command window, type: \\
\noindent \begin{verbatim}
md %homepath%\Version5\
\script\instver5
\end{verbatim}
\end{itemize}
\item  A ``Unix-type" build is possible, provided that Cygwin (see {\tt http://www.cygwin.com/}) is properly installed on your PC. Cygwin is a
native implementation of a {\sl Unix shell}. After installation of
Cygwin, you obtain a native Unix
terminal window where all Version5 components can be executed. The DRAGON installation must proceed
in the Cygwin shell.
\end{itemize}

\vskip 0.08cm

Directory {\tt libraries} contains open-source Draglibs that can be used to
test your implementation. This directory must me installed as shown in \Fig{tree2}
{\sl before} following the instructions of the {\tt readme} file for executing
multigroup tests. The script named
{\tt tdraglib.access} is automatically executed by the {\tt rdragon} script when
no other {\tt *.access} script is provided. The {\tt tdraglib.access} script creates
a symbolic link between the draglib {\tt draglibJef2p2} and file {\tt DLIB\_J2}
used in the {\tt tdraglib.x2m} and {\tt trowland.x2m} non-regression tests.

\vskip 0.08cm

The content of the {\tt readme} file follows:

\listfile{readme}
