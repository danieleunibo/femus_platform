\clearpage
$ $
\vskip 2.0cm

\begin{center}

SUMMARY

\end{center}

The computer code DRAGON contains a collection of models which can simulate the
neutronic behaviour of a unit cell or a fuel assembly in a nuclear reactor. It
includes all of the functions that characterize a lattice cell code, namely: the
interpolation of microscopic cross sections  which are supplied by means of
standard libraries; resonance self-shielding calculations in multidimensional
geometries; multigroup and multidimensional neutron flux calculations which can
take into account neutron leakage; transport-transport or transport-diffusion
equivalence calculations as well as editing of condensed and homogenized nuclear
properties for reactor calculations; and finally isotopic depletion calculations.

\vskip 0.15cm

The code DRAGON contains a multigroup iterator conceived to control a number of
different algorithms for the solution of the neutron transport equation. Each of
these algorithms is presented in the form of a one-group solution procedure
where the contributions from other energy groups are included in a source term.
The current version of DRAGON contains many such algorithms. The
SYBIL option which solves the integral transport equation using the collision
probability method for simple one-dimensional (1--D) geometries (either plane,
cylindrical or spherical) and the interface current method for 2--D Cartesian or hexagonal
assemblies. The EXCELL option which solves the integral transport equation
using the collision probability method for general 2--D geometries and for
three-dimensional (3--D) assemblies. The MCCG option solves the integro-differential
transport equation using the long characteristics method for general 2--D and
3--D geometries.

\vskip 0.15cm

The execution of DRAGON is controlled by the generalized GAN driver. It is
modular and can be interfaced easily with other production codes.
