\subsubsection{The {\tt SNT:} tracking module}\label{sect:SNData}

The {\tt SNT:} module can process one-dimensional, two-dimensional regular geometries and three-dimensional Cartesian geometries
of type \moc{CAR1D}, \moc{TUBE}, \moc{SPHERE}, \moc{CAR2D}, \moc{TUBEZ} and \moc{CAR3D}.

\vskip 0.2cm

The calling specification for this module is:

\begin{DataStructure}{Structure \dstr{SNT:}}
\dusa{TRKNAM}
\moc{:=} \moc{SNT:} $[$ \dusa{TRKNAM} $]$ 
\dusa{GEONAM} \moc{::}  \dstr{desctrack} \dstr{descsn}
\end{DataStructure}

\noindent  where
\begin{ListeDeDescription}{mmmmmmm}

\item[\dusa{TRKNAM}] {\tt character*12} name of the \dds{tracking} data
structure that will contain region volume and surface area vectors in
addition to region identification pointers and other tracking information.
If \dusa{TRKNAM} also appears on the RHS, the previous tracking 
parameters will be applied by default on the current geometry.

\item[\dusa{GEONAM}] {\tt character*12} name of the \dds{geometry} data
structure.

\item[\dstr{desctrack}] structure describing the general tracking data (see
\Sect{TRKData})

\item[\dstr{descsn}] structure describing the transport tracking data
specific to \moc{SNT:}.

\end{ListeDeDescription}

\vskip 0.2cm

The \moc{SNT:} specific tracking data in \dstr{descsn} is defined as

\begin{DataStructure}{Structure \dstr{descsn}}
$[$ \moc{DIAM} \dusa{m} $]$ \\
\moc{SN} \dusa{n} $~[$ \moc{SCAT} \dusa{iscat} $]~~[~\{$ \moc{DSA} $|$ \moc{NDSA} $\}~]
~~[~\{$ \moc{LIVO} \dusa{icl1} \dusa{icl2} $|$ \moc{NLIVO} $\}~]$\\
$[$ \moc{GMRES} \dusa{nstart} $]~[$ \moc{NSDSA} \dusa{nsdsa} $]~
[$ \moc{MAXI} \dusa{maxi} $]~[$  \moc{EPSI} \dusa{epsi} $]~
[$ \moc{QUAD} \dusa{iquad} $]$ \\
$[~[$ \moc{QUAB} \dusa{iquab} $]~[~\{$ \moc{SAPO} $|$ \moc{HEBE} $\}~]~]$ \\
{\tt ;}
\end{DataStructure}

\noindent where

\begin{ListeDeDescription}{mmmmmmm}

\item[\dstr{desctrack}] structure describing the general tracking data (see
\Sect{TRKData})

\item[\moc{DIAM}] keyword to fix the spatial approximation order.

\item[\dusa{m}] spatial order. \dusa{m} $=1$ is used for the classical diamond scheme (default value). \dusa{m} $=2$
or $=3$ is currently available in 1D slab, 2D Cartesian and 3D Cartesian geometries.

\item[\moc{SN}] keyword to fix the angular approximation order of the flux.

\item[\dusa{n}] order of the $S_N$ approximation (even number).

\item[\moc{SCAT}] keyword to limit the anisotropy of scattering sources.

\item[\dusa{iscat}] number of terms in the scattering sources. \dusa{iscat} $=1$ is used for
isotropic scattering in the laboratory system. \dusa{iscat} $=2$ is used for
linearly anisotropic scattering in the laboratory system. The default value is set to $n$.

\item[\moc{LIVO}] keyword to enable Livolant acceleration method (default value).
\item[\dusa{icl1},~\dusa{icl2}] Numbers of respectively free and accerated iterations in the Livolant method.
\item[\moc{NLIVO}] keyword to disable Livolant acceleration method.
\item[\moc{DSA}] keyword to enable diffusion synthetic acceleration using BIVAC or TRIVAC (default value).

\item[\moc{NDSA}] keyword to disable diffusion synthetic acceleration.

\item[\moc{GMRES}] keyword to set the GMRES(m) acceleration of the scattering iterations. The default value,
equivalent to \dusa{nstart}=0, corresponds to a one-parameter Livolant acceleration.\cite{gmres}

\item[\dusa{nstart}] restarts the GMRES method every \dusa{nstart} iterations.

\item[\moc{NSDSA}] keyword to set the number if inner flux iterations {\sl without} DSA in 3D
cases if \dusa{m}~$\ge 2$. If DSA is enabled too soon, instabilities and convergence failure can occur in these cases.

\item[\dusa{nsdsa}] number if inner flux iterations {\sl without} DSA. The default value is \dusa{nsdsa}~$=10$.

\item[\moc{MAXI}] Keyword to set the maximum number of inner iterations (or GMRES iterations if actived).
\item[\dusa{maxi}] Maximum number of inner iterations. Default value: $100$.

\item[\moc{EPSI}] Set the convergence criterion on inner iterations (or GMRES iterations if actived).
\item[\dusa{epsi}] Convergence criterion on inner iterations. The default value is $1\times 10^{-5}$.
\item[\moc{QUAD}] keyword to set the type of angular quadrature.

\item[\dusa{iquad}] type of quadrature: $=1$: Lathrop-Carlson level-symmetric quadrature;
$=2$: $\mu_1$--optimi\-zed level-symmetric quadrature (default option in 2D and in 3D); $=3$ Snow-code level-symmetric quadrature
(obsolete); $=4$: Legendre-Chebyshev quadrature (variable number of base points
per axial level); $=5$: symmetric Legendre-Chebyshev quadrature; $=6$: quadruple range (QR)
quadrature;\cite{quadrupole} $=10$: product of Gauss-Legendre and Gauss-Chebyshev quadrature (equal
number of base points per axial level).

\item[\moc{QUAB}] keyword to specify the number of basis point for the
numerical integration of each micro-structure in cases involving double
heterogeneity (Bihet).

\item[\dusa{iquab}] the number of basis point for the numerical integration of
the collision probabilities in the micro-volumes using the  Gauss-Jacobi
formula. The values permitted are: 1 to 20, 24, 28, 32 or  64. The default value
is \dusa{iquab}=5.

\item[\moc{SAPO}] use the Sanchez-Pomraning double-heterogeneity model.\cite{sapo}

\item[\moc{HEBE}] use the Hebert double-heterogeneity model (default option).\cite{BIHET}

\end{ListeDeDescription}

\eject
