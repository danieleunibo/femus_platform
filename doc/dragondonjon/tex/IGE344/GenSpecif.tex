\subsection{Syntactic rules for input specification}\label{sect:spec}

\vskip 0.2cm
The input data to any module is read in free format using the subroutine 
{\tt REDGET}. CLE-2000 variables\cite{cle2000,gan2} are also allowed.
The user guide for DONJON is written using the following convention:

\begin{itemize}

\item the parameters surrounded by single square brackets `$[\;]$' 
denote an optional input;

\item the parameters surrounded by double square brackets `$[[\;]]$' 
denote an input which may be repeated as many times as needed;

\item the parameters in braces separated by vertical bars `$\{\; |\; |\; \}$' 
denote a choice where one and {\sl only} one input is mandatory;

\item the parameters in {\bf{bold face}} and in brackets `( )' 
denote an input structure;

\item the parameters in italics and in brackets with an index 
`({\it data}(i) ,  i = 1, n )' denote a set of n inputs;

\item the words using the typewriter font {\tt KEYWORD} are
character constants used as keywords;

\item the words in italics denote the user-defined variables:
they are lower-case and of integer type (when starting from
{\it i} to {\it n}), or of real type (when starting from {\it a} to {\it h} or from
{\it o} to {\it z}); or they are upper-case and of character type {\it CHARACTER}.

\end{itemize}
