\section{GENERAL CORE-DESCRIPTION MODULES}\label{sect:modesc1}

\input{SectRESINI}
\vskip 1.0cm
\input{SectUSPLIT}
\vskip 1.0cm
\input{SectMACINI}
\vskip 1.0cm
\input{SectDEVINI}
\vskip 1.0cm
\input{SectDETINI}
\vskip 1.0cm
\input{SectLZC}
\vskip 1.0cm
\input{SectDSET}
\vskip 1.0cm
\input{SectMCC}
\vskip 1.0cm
\input{SectMOVDEV}
\vskip 1.0cm
\input{SectNEWMAC}
\vskip 1.0cm
\input{SectFLPOW}
\vskip 1.0cm
\input{SectTAVG}
\vskip 1.0cm
\input{SectTINST}
\vskip 1.0cm
\input{SectSIM}
\vskip 1.0cm
\input{SectXENON}
\vskip 1.0cm
\input{SectDETECT}
\vskip 1.0cm
\input{SectCVR}
\vskip 1.0cm
\input{SectHST}

\section{CROSS-SECTION INTERPOLATION MODULES}\label{sect:modesc2}

\input{SectCRE}
\vskip 1.0cm
\input{SectNCR}
\vskip 1.0cm
\input{SectSCR}
\vskip 1.0cm
\input{SectAFM}
\vskip 1.0cm
\input{SectT16CPO}
\vskip 1.0cm
\input{SectD2P}
\vskip 1.0cm

\section{THERMAL-HYDRAULICS MODULES}\label{sect:modesc3}

\subsection{The \moc{THM:} module}\label{sect:thm}

\vskip 0.2cm
The \moc{THM:} module is a simplified thermal-hydraulics module where the
reactor is represented as a collection of independent channels with no
cross-flow between them. Each channel is represented using 1D convection
equations along the channel and 1D cylindrical equations for a single pin cell.
A two-fluid homogeneous model is used. The \moc{THM:} module is built around {\sl freesteam},
an open source implementation of IAPWS-IF97 steam tables for light water.\cite{freesteam}.
The \moc{THM:} module works both in steady-state and in transient conditions and includes
a subcooled flow boiling model based on the {\sl Jens \& Lottes correlation} \cite{jenslottes} and on {\sl Bowring's model} for two-phase homogeneous flows \cite{bowring}.

\vskip 0.08cm

The 1D thermal-hydraulics equations are solved in each channel as a fonction of two
{\sl fixed} inlet conditions for the coolant velocity and temperature and one {\sl fixed} outlet condition for the pressure.

\vskip 0.1cm

\noindent
The \moc{THM:} module specification is:

\begin{DataStructure}{Structure \moc{THM:}}
\dusa{THERMO} \dusa{MAPFL} \moc{:=} \moc{THM:}
$[$ \dusa{THERMO} $]$ \dusa{MAPFL} \moc{::} \dstr{descthm}
\end{DataStructure}

\noindent where

\begin{ListeDeDescription}{mmmmmmmm}

\item[\dusa{THERMO}] \texttt{character*12} name of the \dds{thermo}
object that will be created or updated by the \moc{THM:} module. Object \dds{thermo}
contains thermal-hydraulics information set or computed by \moc{THM:} in transient or in
permanent conditions such as the distribution of the enthalpy, the pressure, the velocity,
the density and the temperatures of the coolant for all the channels in the geometry. It also contains all the values of the fuel temperatures in transient or in permanent conditions according to the discretisation chosen for the fuel rods.

\item[\dusa{MAPFL}] \texttt{character*12} name of the \dds{map} 
object containing fuel regions description and local parameter informations.

\item[\dstr{descthm}] structure describing the input data to the \moc{THM:} module. 

\end{ListeDeDescription}

\vskip 0.2cm

\subsubsection{Input data to the \moc{THM:} module}\label{sect:thmstr}

\begin{DataStructure}{Structure \dstr{descthm}}
$[$ \moc{EDIT} \dusa{iprint} $]$ \\
$[$ \moc{RELAX} \dusa{relax} $]$ \\
$[$ \moc{TIME} \dusa{caltype} \dusa{timestep} \dusa{timeiter} \dusa{time} $]$ \\
$[$ \moc{FPUISS} \dusa{fract} $]~[$ \moc{CRITFL} \dusa{cflux} $]$ \\
$\{$ \moc{CWSECT} \dusa{sect} \dusa{flow} $|$ \moc{SPEED} \dusa{velocity} $\}$ \\
\moc{ASSMB} \dusa{sass} \dusa{nbf} \dusa{nbg} \\
\moc{INLET} \dusa{poutlet} \dusa{tinlet} \\
\moc{RADIUS} \dusa{r1} \dusa{r2} \dusa{r3} \dusa{r4} \\
$\{~[$ \moc{POROS} \dusa{poros} $]~[$ \moc{PUFR} \dusa{pufr} $]~|~[$ \moc{CONDF} \dusa{ncond} (\dusa{kcond}(k),k=0,\dusa{ncond}) $[$ \moc{INV} \dusa{inv} \dusa{ref} $]$ \dusa{unit} $]~\}$ \\
$[$ \moc{CONDC} \dusa{ncond} (\dusa{kcond}(k),k=0,\dusa{ncond}) \dusa{unit} $]$ \\
$[$ \moc{HGAP} \dusa{hgap} $]~[$ \moc{HCONV} \dusa{hconv} $]~[$ \moc{TEFF} \dusa{wteff} $]$ \\
$[$ \moc{CONV} \dusa{maxit1} \dusa{maxit2} \dusa{maxit3} \dusa{ermaxt} \dusa{ermaxc} $]$ \\
$[$ \moc{RODMESH} \dusa{nb1} \dusa{nb2} $]$ \\
$[$ \moc{FORCEAVE} $]$ \\
$[~\{$ \moc{BOWR} $|$ \moc{SAHA} $\}~]$ \\
$[[$ \moc{SET-PARAM} \dusa{PNAME} \dusa{pvalue} $]]$ \\
;
\end{DataStructure}

\noindent where
\begin{ListeDeDescription}{mmmmmmmm}

\item[\moc{EDIT}] keyword used to set \dusa{iprint}.

\item[\dusa{iprint}] integer index used to control the printing on screen:
= 0 for no print; = 1 for minimum printing; larger values produce
increasing amounts of output.

\item[\moc{RELAX}] keyword used to set the relaxation parameter \dusa{relax}.

\item[\dusa{relax}] relaxation parameter selected in the interval $0<$\dusa{relax}$\le 1$ and used to update
the fuel (average and surface) temperature, coolant temperature and coolant density. The updated value is taken equal to
$(1-$\dusa{relax}$)$ times the previous iteration value plus \dusa{relax} times the actual iteration value. The default
value is \dusa{relax}$=1$.

\item[\moc{TIME}] keyword used to specify the type of calculation (steady-state or transient) performed by the \moc{THM:} module and the temporal parameters in case of a transient calculation. By
default, a steady-state calculation is performed.
\item[\dusa{caltype}] integer value set to control the type of calculation that will be performed by the \moc{THM:} module: =0 for steady-state; =1 for transient. The default value is 0.
\item[\dusa{timestep}] real value set to the time step in case of a transient calculation. The default value is 0.0.
\item[\dusa{timeiter}] integer value of the current time step index, used for transient calculations. The default value is 0.
\item[\dusa{time}] real value of time in second, used for transient calculations. The default value is 0.0.

\item[\moc{FPUISS}] keyword used to specify the fraction of the power released in fuel. The remaining
fraction is assumed to be released in coolant. The default value is 0.974.

\item[\dusa{fract}] real value set to the fraction ($f$). Power densities released in coolant and
fuel are computed as
\begin{eqnarray*}
Q_{\rm cool} \negthinspace\negthinspace &=& \negthinspace\negthinspace (1-f)\, {V_{\rm cool}+V_{\rm fuel} \over V_{\rm cool}} \, {P_{\rm mesh} \over
V_{\rm mesh}} \\
Q_{\rm fuel} \negthinspace\negthinspace &=& \negthinspace\negthinspace f\, {V_{\rm cool}+V_{\rm fuel} \over V_{\rm fuel}} \, {P_{\rm mesh} \over
V_{\rm mesh}}
\end{eqnarray*}
\noindent where $V_{\rm cool}$ and $V_{\rm fuel}$ are coolant and fuel area computed from
\dusa{sass}, \dusa{nbf}, \dusa{nbg}, \dusa{r3} and \dusa{r4}. The mesh power $P_{\rm mesh}$ and
volume $V_{\rm mesh}$ are recovered from \dusa{MAPFL} object.

\item[\moc{CRITFL}] keyword used to specify the critical heat flux.

\item[\dusa{cflux}] real value set to the critical heat flux in W/m$^2$. The default value is 2.0
$\times$ 10$^6$ W/m$^2$.

\item[\moc{CWSECT}] keyword used to specify the core coolant section and the coolant inlet flow.

\item[\dusa{sect}] real value set to the core coolant section in m$^2$.

\item[\dusa{flow}] real value set to the coolant flow in m$^3$/hr. This value doesn't include the by-pass flow.
The inlet coolant velocity in m/s is computed as $$V={{\sl flow} \over 3600 \ {\sl cwsect}}.$$

\item[\moc{SPEED}] keyword used to specify the inlet coolant velocity.

\item[\dusa{velocity}] real value set to the inlet coolant velocity in m/s.

\item[\moc{ASSMB}] keyword used to specify the assembly characteristics.

\item[\dusa{sass}] real value set to the assembly surface in m$^2$. This value is equal to the square of
an assembly side (including the water gap).

\item[\dusa{nbf}] integer value set to the number of active fuel rods in a single assembly.

\item[\dusa{nbg}] integer value set to the number of active guide tubes in a single assembly.

\item[\moc{INLET}] keyword used to specify the outlet pressure and inlet absolute temperature.

\item[\dusa{poutlet}] real value set to the outlet coolant pressure in Pa. The pressure along each channel is assumed to be
constant and equal to \dusa{poutlet} in permanent conditions.

\item[\dusa{tinlet}] real value set to the inlet coolant absolute temperature in K.

\item[\moc{RADIUS}] keyword used to set the pin-cell radii.

\item[\dusa{r1}] real value set to the fuel pellet radius in m.

\item[\dusa{r2}] real value set to the internal clad rod radius in m.

\item[\dusa{r3}] real value set to the external clad rod radius in m.

\item[\dusa{r4}] real value set to the guide tube radius in m.

\item[\moc{POROS}] keyword used to set the oxyde porosity of fuel. Porosity affects some built-in correlations
used to represent the heat conduction phenomenon in fuel.

\item[\dusa{poros}] real value set to the oxyde porosity. The default value is 0.05.

\item[\moc{PUFR}] keyword used to set the plutonium mass enrichment of fuel. Plutonium enrichment affects some built-in correlations
used to represent the heat conduction phenomenon in fuel.

\item[\dusa{pufr}] real value set to the plutonium mass enrichment. The default value is 0.0.

\item[\moc{CONDF}] keyword used to set the fuel thermal conductivity as a function of local fuel temperature $T_{fuel}$.
Fuel conductivity is computed as

$$\lambda_{fuel} = \sum_{k=0}^{\dusa{ncond}} {\dusa{kcond}(k)*(T_{fuel})^k + \frac{\dusa{inv}}{T_{fuel}-\dusa{ref}}}$$

with $\lambda_{fuel}$ in $W/m/K$ and $T_{fuel}$ in the selected unit (Kelvin or Celsius).

By default, built-in models are used, taking into account oxyde porosity and plutonium mass enrichment.
Note that oxyde porosity and plutonium mass enrichment are ignored if this keyword is used.

\item[\dusa{ncond}] integer value set to the degree of the conductivity polynomial.

\item[\dusa{kcond}] real value set to the coefficient of the conductivity polynomial. $\dusa{ncond}+1$ coefficients are expected.

\item[\dusa{unit}] string value set to the unit of temperature $T$ in the conductivity function. Can be either \dusa{CELSIUS} or \dusa{KELVIN}.

\item[\moc{INV}] keyword used to add an inverse term in the fuel conductivity function.

\item[\dusa{inv}] real value set to the coefficient in the inverse term of fuel conductivity.
The default value is 0.0 (i.e. no inverse term).

\item[\dusa{ref}] real value set to the reference in the inverse term of fuel conductivity.

\item[\moc{CONDC}] keyword used to set the clad thermal conductivity as a function of local clad temperature $T_{clad}$.
Clad conductivity is computed with the following polynomial

$$\lambda_{clad} = \sum_{k=0}^{\dusa{ncond}} {\dusa{kcond}(k)*(T_{clad})^k}$$

with $\lambda_{clad}$ in $W/m/K$ and $T_{clad}$ in the selected unit (Kelvin or Celsius).

By default, a built-in model is used.

\item[\moc{HGAP}] keyword used to set the heat exchange coefficient of the gap as a constant.
By default, a built-in model is used.

\item[\dusa{hgap}] real value set to the constant heat exchange coefficient of the gap in $W/m^2/K$.

\item[\moc{HCONV}] keyword used to set the heat transfer coefficient between clad and fluid as a constant.
By default, this coefficient is computed using a built-in correlation.

\item[\dusa{hconv}] real value set to the constant heat transfer coefficient between clad and fluid in $W/m^2/K$.

\item[\moc{TEFF}] keyword used to set the weighting factor in the effective fuel temperature approximation.
The effective fuel temperature is used for the cross sections interpolations on fuel temperature.

\item[\dusa{wteff}] real value $W_{\rm teff}$ set to the weighting factor in the effective fuel temperature.
The effective fuel temperature is computed as

$$
T^{\rm fuel}_{\rm eff}=W_{\rm teff}*T^{\rm fuel}_{\rm surface}+(1-W_{\rm teff})*T^{\rm fuel}_{\rm center}
$$

where $0\le W_{\rm teff} \le 1$, $T^{\rm fuel}_{\rm surface}$ is the temperature at the surface of the fuel pellet (K), and $T^{\rm fuel}_{\rm center}$ is the temperature at the center of the fuel pellet (K).

By default, the Rowlands weighting factor $W_{\rm teff}={5 \over 9}$ is used\cite{Rowlands}.

\item[\moc{CONV}] keyword used to set the convergence criteria for solving the conduction and the conservation equation.

\item[\dusa{maxit1}] integer value set to the maximum number of iterations for computing the
conduction integral. The default value is 50.

\item[\dusa{maxit2}] integer value set to the maximum number of iterations for computing the
center pellet temperature. The default value is 50.

\item[\dusa{maxit3}] integer value set to the maximum number of iterations for computing the
coolant parameters (mass flux, pressure, enthalpy and density) in case of a transient calculation. The default value is 50.

\item[\dusa{ermaxt}] real value set to the maximum temperature error in K. The default value is 1~K.

\item[\dusa{ermaxc}] real value set to the maximum relative  error for parameters given by the resolution of flow conservation equations (pressure, velocity and enthalpy). The default value is $10^{-3}$.

\item[\moc{RODMESH}] keyword used to set the radial discretization of pin-cells.

\item[\dusa{nb1}] integer value set to the number of discretisation points in fuel. The default value
is 5.

\item[\dusa{nb2}] integer value set to the number of discretisation points in the whole pin-cell (fuel+cladding). The default value
is 8.

\item[\moc{FORCEAVE}] keyword used to force the use of the average approximation during the fuel conductivity evaluation.
By default, a rectangle quadrature approximation is used.

\item[\moc{BOWR}] keyword used to set a subcooling model based on the Jens \& Lottes correlation\cite{jenslottes} with the Bowring model\cite{bowring} (default option).

\item[\moc{SAHA}] keyword used to set a subcooling model based on the Saha-Zuber correlation\cite{lahey}. This option is recommended for BWR applications.

\item[\moc{SET-PARAM}] keyword used to indicate the input (or modification)
of the actual values for a parameter specified using its \dusa{PNAME}.

\item[\moc{PNAME}] keyword used to specify \dusa{PNAME}.

\item[\dusa{PNAME}] \texttt{character*12} name of a parameter.

\item[\dusa{pvalue}] single real value containing the actual
parameter's values. Note that this value will not be checked for consistency
by the module. It is the user responsibility to provide the valid parameter's value
which should be consistent with those recorded in the multicompo or Saphyb database.

\end{ListeDeDescription}
\clearpage


\section{OPTIMIZATION MODULES}\label{sect:modesc4}

This section is related to optimization capabilities available in Donjon and
based on generalized perturbation theory.\cite{optex1,optex2} General information
about the generalized perturbation theory can be found in Sect. 5.3 of Ref.~\citen{PIP2009}.

\input{SectDLEAK}
\vskip 1.0cm
\subsection{The \texttt{GRAD:} module}

The {\tt GRAD:} module is designed to perform the following tasks:
\begin{itemize}
\item compute the gradients of the {\sl system characteristics} using solutions of direct or adjoint
fixed source eigenvalue problems. Here, we assume an optimization problem with \dusa{nvar} control variables and
with \dusa{ncst} constraints. The total number of system characteristics is therefore equal to \dusa{ncst}$+1$.

\item define options and parameters for the different method to solve the optimization problem. The non-linear
optimization problem can be solved as a converging sequence of linear optimization problems with a quadratic constraint
of the form
$$
\sum_{i=1}^{\sl nvar} \omega_i \left( \Delta x_i^{(n)}\right)^2 \le \left( S^{(n)}\right)^2
$$
\noindent where $\omega_i$ is a weight defined after keyword \moc{CST-WEIGHT} and $\Delta x_i^{(n)}$ is a displacement for
$i$--th control variable at iteration $(n)$. The initial value of radius $S^{(1)}$ is defined after keyword \moc{OUT-STEP-LIM}.

\item reduces the radius $S^{(n)}$ of the quadratic constraint.
\end{itemize}

\vskip 0.08cm

The calling specifications are:

\begin{DataStructure}{Structure \moc{GRAD:}}
\dusa{OPTIM} \moc{:=} \moc{GRAD:} $[$ \dusa{OPTIM} $]$ \dusa{DFLUX} \dusa{GPT} \moc{::} \dstr{grad\_data}
\end{DataStructure}

\noindent where

\begin{ListeDeDescription}{mmmmmmmm}

\item[\dusa{OPTIM}] \texttt{character*12} name of the \dds{optimize} object ({\tt L\_OPTIMIZE} signature) containing the
optimization informations. Object \dusa{OPTIM} must appear on the RHS to be able to updated the previous values.

\item[\dusa{DFLUX}] \texttt{character*12} name of the \dds{flux} object ({\tt L\_FLUX} signature) containing a set of
solutions of fixed-source eigenvalue problems.

\item[\dusa{GPT}] \texttt{character*12} name of the \dds{gpt} object ({\tt L\_GPT} signature) containing a set of
direct or adjoint sources.

\item[\dstr{grad\_data}] structure containing the data to the module \texttt{GRAD:} (see Sect.~\ref{sect:grad_data}).

\end{ListeDeDescription}
\vskip 0.2cm

\subsubsection{Data input for module \texttt{GRAD:}}\label{sect:grad_data}

\begin{DataStructure}{Structure \moc{grad\_data}}
$[$ \moc{EDIT} \dusa{iprint} $]$ \\
$[$ \moc{METHOD} \{ \moc{SIMPLEX} $|$ \moc{LEMKE} $|$ \moc{MAP} $|$ \moc{AUG-LAGRANG} $|$ \moc{PENAL-METH} \} $]$ \\
$[$ \moc{OUT-STEP-LIM} \dusa{sr} $]$ \\
$[$ \moc{OUT-STEP-EPS} \dusa{$\epsilon_{ext}$} $]~[$ \moc{INN-STEP-EPS} \dusa{$\epsilon_{inn}$} $]$ \\
$[$ \moc{CST-QUAD-EPS} \dusa{$\epsilon_{quad}$} $]$ \\
$[~\{$ \moc{MAXIMIZE} $|$ \moc{MINIMIZE} $\}~]$ \\
$[$ \moc{STEP-REDUCT} $\{$ \moc{HALF} $|$ \moc{PARABOLIC} $\}~]$ \\
$[$ \moc{VAR-VALUE} ( \dusa{control}(i), i=1,\dusa{nvar} ) $]~[$ \moc{VAR-WEIGHT} ( \dusa{weight}(i), i=1,\dusa{nvar} ) $]$ \\
$[$ \moc{VAR-VAL-MIN} $\{$ ( \dusa{vecmin}(i), i=1,\dusa{nvar} ) $|$ \moc{ALL} \dusa{varmin} $]$ \\
$[$ \moc{VAR-VAL-MAX} $\{$ ( \dusa{vecmax}(i), i=1,\dusa{nvar} ) $|$ \moc{ALL} \dusa{varmax} $]$ \\
$[$ \moc{FOBJ-CST-VAL} ( \dusa{funct}(i), i=1,\dusa{ncst}$+1$ ) $]$ \\
$[$ \moc{CST-TYPE} ( \dusa{type}(i), i=1,\dusa{ncst} ) $]~[$ \moc{CST-OBJ} ( \dusa{cstval}(i), i=1,\dusa{ncst} ) $]$ \\
$[$ \moc{CST-WEIGHT} ( \dusa{cstw}(i), i=1,\dusa{ncst} ) $]$ \\
;
\end{DataStructure}

\noindent where
\begin{ListeDeDescription}{mmmmmmmm}

\item[\moc{EDIT}] keyword used to set \dusa{iprint}.

\item[\dusa{iprint}] index used to control the printing in module.

\item[\moc{METHOD}] keyword used to define the quasi-linear programming method. {\bf Note:} If the general Lemke method is
used, the quadratic constraint must be active. The strategy consists to proceed in two steps:
\begin{itemize}
\item At first step, the linear programming problem
(i. e., without the quadratic contraint) is solved and the control-variable displacement is computed. If this displacement is less
than the radius of the quadratic constraint, the step one solution is accepted and step two is not performed. If this displacement is greater
than the radius of the quadratic constraint, the step one solution is rejected and step two is performed. Step one can be
solved with the SIMPLEX method or with the linear LEMKE method.
\item At step two, the general LEMKE method is used to find the correct solution. The general Lemke method is based on a parametric linear
complementarity principle, as explained in Ref.~\citen{ferland}.
\end{itemize}

\item[\moc{SIMPLEX}] keyword used to specify that the SIMPLEX method will be used at step one and the general LEMKE method at step two.

\item[\moc{LEMKE}] keyword used to specify that the linear LEMKE method will be used at step one and the general LEMKE method at step two.

\item[\moc{MAP}] keyword used to specify that the MAP method will be used. The quadratic constraint is linearized and a converging sequence
of SIMPLEX calculations is performed.

\item[\moc{AUG-LAGRANG}] keyword used to specify that the augmented Lagrangian method will be used.

\item[\moc{PENAL-METH}] keyword used to specify that the penalty method will be used.

\item[\moc{OUT-STEP-LIM}] keyword used to set the initial radius of the quadratic constraint (default value is \dusa{sr} $=1.0$).

\item[\dusa{sr}] initial radius of the quadratic constraint (real).

\item[\moc{OUT-STEP-EPS}] keyword used to set the tolerance of outer iteration convergence inside module {\tt PLQ:}.

\item[\dusa{$\epsilon_{ext}$}] tolerance value (real).

\item[\moc{INN-STEP-EPS}] keyword used to set the tolerance used within the SIMPLEX or LEMKE method.

\item[\dusa{$\epsilon_{inn}$}] tolerance value (real).

\item[\moc{CST-QUAD-EPS}] keyword to set the convergence parameter \dusa{epsilon4} for the radius of the quadratic constraint inside module {\tt GRAD:}.

\item[\dusa{$\epsilon_{quad}$}] tolerance for convergence of the radius of the quadratic constraint (real).

\item[\moc{MAXIMIZE}] keyword used to specify that the optimization problem will be a maximization.

\item[\moc{MINIMIZE}] keyword used to specify that the optimization problem will be a minimization (default).

\item[\moc{STEP-REDUCT}] keyword used to define the method of the reduction of the outer step.

\item[\moc{HALF}] keyword used to specify that the step will be reduced by a factor of 2.

\item[\moc{PARABOLIC}] keyword used to specify that the step will be reduced with the parabolic method.

\item[\moc{VAR-VALUE}] keyword to specify the values of the control variables. These values can also be set in a previous call
to module {\tt GRAD:} or set in another module.

\item[\dusa{control}] array containing \dusa{nvar} real values.

\item[\moc{VAR-WEIGHT}] keyword to specify the values of the control variable weights in the quadratic constraint. All weights
are set to 1.0 by default.

\item[\dusa{weight}] array containing \dusa{nvar} real values.

\item[\moc{VAR-VAL-MIN}] keyword to specify the minimum values of the control variables. These values can also be set in a previous call
to module {\tt GRAD:}.

\item[\dusa{vecmin}] array containing \dusa{nvar} real values.

\item[\dusa{varmin}] single real value used for all control variables.

\item[\moc{VAR-VAL-MAX}] keyword to specify the maximum values of the control variables. These values can also be set in a previous call
to module {\tt GRAD:}.

\item[\dusa{vecmax}] array containing \dusa{nvar} real values.

\item[\dusa{varmax}] single real value used for all control variables.

\item[\moc{FOBJ-CST-VAL}] keyword to specify the value of the objective function followed by the actual values of the constraints. These values can also be set in a previous call
to module {\tt GRAD:} or set in another module.

\item[\dusa{funct}] array containing \dusa{ncst}$+1$ real values.

\item[\moc{CST-TYPE}] keyword to specify the relation types of the constraints. These values can also be set in a previous call
to module {\tt GRAD:}.

\item[\dusa{type}] array containing \dusa{ncst} integer values. These values are: $=-1$ for $\ge$, $=0$ for equalily and $=1$
for $\le$.

\item[\moc{CST-OBJ}] keyword to specify the RHS values of the constraints. These values can also be set in a previous call
to module {\tt GRAD:}.

\item[\dusa{cstval}] array containing \dusa{ncst} real values.

\item[\moc{CST-WEIGHT}] keyword to specify the weights (or penalties) of the constraints. These weights are not used with
Lemke or MAP methods. These values can also be set in a previous call to module {\tt GRAD:}.

\item[\dusa{cstw}] array containing \dusa{ncst} real values.

\end{ListeDeDescription}
\clearpage

\vskip 1.0cm
\subsection{The \texttt{PLQ:} module}

The {\tt PLQ:} module is used to solve the linear programming problem with a quadratic constraint. The gradients of the {\sl system
characteristics} are calculated with module {\tt GRAD:}. The options and parameters for the different method to solve the optimization problem
are also defined in module {\tt GRAD:}.

\vskip 0.08cm

The calling specifications are:

\begin{DataStructure}{Structure \moc{PLQ:}}
\dusa{OPTIM} \moc{:=} \moc{PLQ:} $[$ \dusa{OPTIM} $]$ \moc{::} \dstr{plq\_data}
\end{DataStructure}

\noindent where

\begin{ListeDeDescription}{mmmmmmmm}

\item[\dusa{OPTIM}] \texttt{character*12} name of the \dds{optimize} object ({\tt L\_OPTIMIZE} signature) containing the
optimization informations. Object \dusa{OPTIM} must appear on the RHS to be able to updated the previous values.

\item[\dstr{plq\_data}] structure containing the data to the module \texttt{PLQ:} (see Sect.~\ref{sect:plq_data}).

\end{ListeDeDescription}
\vskip 0.2cm

\subsubsection{Data input for module \texttt{PLQ:}}\label{sect:plq_data}

\begin{DataStructure}{Structure \moc{plq\_data}}
$[$ \moc{EDIT} \dusa{iprint} $]$ \\
$[$ \moc{WARNING-ONLY} $]$ \\
\moc{CALCUL-DX} $[$ \moc{NO-STORE-OLD} $]$ \\
$[$ \moc{COST-EXTRAP} {\tt >>} \dusa{ecost} {\tt <<} $]$ \\
$[$ \moc{CONV-TEST} {\tt >>} \dusa{$l_{conv}$} {\tt <<} $]$ \\
;
\end{DataStructure}

\noindent where
\begin{ListeDeDescription}{mmmmmmmm}

\item[\moc{EDIT}] keyword used to set \dusa{iprint}.

\item[\dusa{iprint}] index used to control the printing in module.

\item[\moc{WARNING-ONLY}] keyword used to specify that only a warning will be used when no valid previous decision vectors can
be recall in case of error of the mathematical programming.

\item[\moc{CALCUL-DX}] keyword used to specify that the new step will be calculated.

\item[\moc{NO-STORE-OLD}] keyword used to specify that the old value of decision variables and gradients will not be stored in
the {\tt L\_OPTIMIZE/'OLD-VALUE'} directory.

\item[\moc{COST-EXTRAP}] keyword used to calculate the extrapolated objective constant \dusa{ecost}.

\item[\dusa{ecost}] extrapolated objective constant.

\item[\moc{CONV-TEST}] keyword used to calculate if the external convergence has been reached.

\item[\dusa{$l_{conv}$}] $=1$ means that external convergence has been reached; $=0$ otherwise.

\end{ListeDeDescription}
\clearpage


\section{PIN-POWER RECONSTRUCTION MODULES}\label{sect:modesc5}

This section is related to pin-power reconstruction capabilities available in Donjon.
The corresponding theory is explained in \cite{Chambon2014,Fliscounakis2011} 

\input{SectNAP}

